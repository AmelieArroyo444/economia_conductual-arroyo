% Options for packages loaded elsewhere
\PassOptionsToPackage{unicode}{hyperref}
\PassOptionsToPackage{hyphens}{url}
\PassOptionsToPackage{dvipsnames,svgnames,x11names}{xcolor}
%
\documentclass[
  letterpaper,
  DIV=11,
  numbers=noendperiod]{scrartcl}

\usepackage{amsmath,amssymb}
\usepackage{lmodern}
\usepackage{iftex}
\ifPDFTeX
  \usepackage[T1]{fontenc}
  \usepackage[utf8]{inputenc}
  \usepackage{textcomp} % provide euro and other symbols
\else % if luatex or xetex
  \usepackage{unicode-math}
  \defaultfontfeatures{Scale=MatchLowercase}
  \defaultfontfeatures[\rmfamily]{Ligatures=TeX,Scale=1}
\fi
% Use upquote if available, for straight quotes in verbatim environments
\IfFileExists{upquote.sty}{\usepackage{upquote}}{}
\IfFileExists{microtype.sty}{% use microtype if available
  \usepackage[]{microtype}
  \UseMicrotypeSet[protrusion]{basicmath} % disable protrusion for tt fonts
}{}
\makeatletter
\@ifundefined{KOMAClassName}{% if non-KOMA class
  \IfFileExists{parskip.sty}{%
    \usepackage{parskip}
  }{% else
    \setlength{\parindent}{0pt}
    \setlength{\parskip}{6pt plus 2pt minus 1pt}}
}{% if KOMA class
  \KOMAoptions{parskip=half}}
\makeatother
\usepackage{xcolor}
\setlength{\emergencystretch}{3em} % prevent overfull lines
\setcounter{secnumdepth}{-\maxdimen} % remove section numbering
% Make \paragraph and \subparagraph free-standing
\ifx\paragraph\undefined\else
  \let\oldparagraph\paragraph
  \renewcommand{\paragraph}[1]{\oldparagraph{#1}\mbox{}}
\fi
\ifx\subparagraph\undefined\else
  \let\oldsubparagraph\subparagraph
  \renewcommand{\subparagraph}[1]{\oldsubparagraph{#1}\mbox{}}
\fi


\providecommand{\tightlist}{%
  \setlength{\itemsep}{0pt}\setlength{\parskip}{0pt}}\usepackage{longtable,booktabs,array}
\usepackage{calc} % for calculating minipage widths
% Correct order of tables after \paragraph or \subparagraph
\usepackage{etoolbox}
\makeatletter
\patchcmd\longtable{\par}{\if@noskipsec\mbox{}\fi\par}{}{}
\makeatother
% Allow footnotes in longtable head/foot
\IfFileExists{footnotehyper.sty}{\usepackage{footnotehyper}}{\usepackage{footnote}}
\makesavenoteenv{longtable}
\usepackage{graphicx}
\makeatletter
\def\maxwidth{\ifdim\Gin@nat@width>\linewidth\linewidth\else\Gin@nat@width\fi}
\def\maxheight{\ifdim\Gin@nat@height>\textheight\textheight\else\Gin@nat@height\fi}
\makeatother
% Scale images if necessary, so that they will not overflow the page
% margins by default, and it is still possible to overwrite the defaults
% using explicit options in \includegraphics[width, height, ...]{}
\setkeys{Gin}{width=\maxwidth,height=\maxheight,keepaspectratio}
% Set default figure placement to htbp
\makeatletter
\def\fps@figure{htbp}
\makeatother

\KOMAoption{captions}{tableheading}
\makeatletter
\makeatother
\makeatletter
\makeatother
\makeatletter
\@ifpackageloaded{caption}{}{\usepackage{caption}}
\AtBeginDocument{%
\ifdefined\contentsname
  \renewcommand*\contentsname{Table of contents}
\else
  \newcommand\contentsname{Table of contents}
\fi
\ifdefined\listfigurename
  \renewcommand*\listfigurename{List of Figures}
\else
  \newcommand\listfigurename{List of Figures}
\fi
\ifdefined\listtablename
  \renewcommand*\listtablename{List of Tables}
\else
  \newcommand\listtablename{List of Tables}
\fi
\ifdefined\figurename
  \renewcommand*\figurename{Figure}
\else
  \newcommand\figurename{Figure}
\fi
\ifdefined\tablename
  \renewcommand*\tablename{Table}
\else
  \newcommand\tablename{Table}
\fi
}
\@ifpackageloaded{float}{}{\usepackage{float}}
\floatstyle{ruled}
\@ifundefined{c@chapter}{\newfloat{codelisting}{h}{lop}}{\newfloat{codelisting}{h}{lop}[chapter]}
\floatname{codelisting}{Listing}
\newcommand*\listoflistings{\listof{codelisting}{List of Listings}}
\makeatother
\makeatletter
\@ifpackageloaded{caption}{}{\usepackage{caption}}
\@ifpackageloaded{subcaption}{}{\usepackage{subcaption}}
\makeatother
\makeatletter
\@ifpackageloaded{tcolorbox}{}{\usepackage[many]{tcolorbox}}
\makeatother
\makeatletter
\@ifundefined{shadecolor}{\definecolor{shadecolor}{rgb}{.97, .97, .97}}
\makeatother
\makeatletter
\makeatother
\ifLuaTeX
  \usepackage{selnolig}  % disable illegal ligatures
\fi
\IfFileExists{bookmark.sty}{\usepackage{bookmark}}{\usepackage{hyperref}}
\IfFileExists{xurl.sty}{\usepackage{xurl}}{} % add URL line breaks if available
\urlstyle{same} % disable monospaced font for URLs
\hypersetup{
  pdftitle={Politicas\_rl},
  colorlinks=true,
  linkcolor={blue},
  filecolor={Maroon},
  citecolor={Blue},
  urlcolor={Blue},
  pdfcreator={LaTeX via pandoc}}

\title{Politicas\_rl}
\author{}
\date{}

\begin{document}
\maketitle
\ifdefined\Shaded\renewenvironment{Shaded}{\begin{tcolorbox}[boxrule=0pt, frame hidden, interior hidden, enhanced, breakable, borderline west={3pt}{0pt}{shadecolor}, sharp corners]}{\end{tcolorbox}}\fi

\hypertarget{poluxedtica}{%
\subsubsection{\texorpdfstring{🔁
\textbf{Política}}{🔁 Política}}\label{poluxedtica}}

Una política es la regla que un agente sigue para \textbf{decidir qué
acción tomar} en cada estado. Es clave para \textbf{maximizar
recompensas a largo plazo}. Es como la estrategia del jugador en un
videojuego.

\begin{center}\rule{0.5\linewidth}{0.5pt}\end{center}

\hypertarget{el-problema-del-bandido-de-muxfaltiples-brazos}{%
\subsubsection{\texorpdfstring{🎰 \textbf{El problema del bandido de
múltiples
brazos}}{🎰 El problema del bandido de múltiples brazos}}\label{el-problema-del-bandido-de-muxfaltiples-brazos}}

Imagina una máquina con varios botones (brazos), cada uno con distinta
probabilidad de darte una recompensa. Para saber cuál es el mejor,
necesitas \textbf{probar todos suficientes veces}, pero si
\textbf{exploras demasiado, pierdes tiempo para explotar el mejor}. Ahí
nace\ldots{}

\begin{center}\rule{0.5\linewidth}{0.5pt}\end{center}

\hypertarget{dilema-exploraciuxf3n---explotaciuxf3n}{%
\subsubsection{\texorpdfstring{⚖️ \textbf{Dilema Exploración -
Explotación}}{⚖️ Dilema Exploración - Explotación}}\label{dilema-exploraciuxf3n---explotaciuxf3n}}

Un clásico: ¿probar nuevas opciones (explorar) o seguir con la que ya
conoces que funciona (explotar)?\\

🔹 \textbf{Explorar mucho} = Menos recompensas inmediatas.\\

🔹 \textbf{Explotar rápido} = Riesgo de perder mejores opciones.

\begin{center}\rule{0.5\linewidth}{0.5pt}\end{center}

\hypertarget{valores-q}{%
\subsubsection{\texorpdfstring{🧠 \textbf{Valores
Q}}{🧠 Valores Q}}\label{valores-q}}

Son las ``expectativas de recompensa'' por tomar una acción en un
estado. Pero \textbf{saber el valor Q no basta}: el agente debe
convertir esos valores en acciones reales mediante una \textbf{regla de
elección}.

\begin{center}\rule{0.5\linewidth}{0.5pt}\end{center}

\hypertarget{reglas-de-elecciuxf3n}{%
\subsubsection{\texorpdfstring{🎯 \textbf{Reglas de
elección}}{🎯 Reglas de elección}}\label{reglas-de-elecciuxf3n}}

Son fórmulas o algoritmos que \textbf{transforman valores Q en
decisiones}. Te ayudan a decidir entre múltiples opciones.

\begin{center}\rule{0.5\linewidth}{0.5pt}\end{center}

\hypertarget{regla-de-maximizaciuxf3n-greedy}{%
\subsubsection{\texorpdfstring{💎 \textbf{Regla de Maximización
(Greedy)}}{💎 Regla de Maximización (Greedy)}}\label{regla-de-maximizaciuxf3n-greedy}}

Selecciona \textbf{siempre la acción con mayor Q}.\\

❌ Problemas:

\begin{itemize}
\item
\item
  Las personas \textbf{no eligen siempre lo mejor} (variabilidad
  empírica).
\item
\item
  Se puede \textbf{quedar atrapado en un máximo local}, sin explorar
  mejores opciones.
\item
\end{itemize}

\begin{center}\rule{0.5\linewidth}{0.5pt}\end{center}

\hypertarget{regla-uxe9psilon-codiciosa}{%
\subsubsection{\texorpdfstring{⚡ \textbf{Regla
épsilon-codiciosa}}{⚡ Regla épsilon-codiciosa}}\label{regla-uxe9psilon-codiciosa}}

Solución para el dilema exploración/explotación:

\begin{itemize}
\item
\item
  Con \textbf{probabilidad (1 - ε)} elige la mejor acción.
\item
\item
  Con \textbf{probabilidad ε}, elige aleatoriamente.
\item
\end{itemize}

🔢 Ejemplo:

\begin{itemize}
\item
\item
  ε = 0.1: Explora el 10\% del tiempo.
\item
\item
  ε = 0.01: Explora menos, pero puede encontrar mejores resultados a
  largo plazo.
\item
\end{itemize}

\begin{center}\rule{0.5\linewidth}{0.5pt}\end{center}

\hypertarget{funciones-de-respuesta-probabiluxedsticas}{%
\subsubsection{\texorpdfstring{🧪 \textbf{Funciones de respuesta
probabilísticas}}{🧪 Funciones de respuesta probabilísticas}}\label{funciones-de-respuesta-probabiluxedsticas}}

Modelos de psicología que \textbf{explican elecciones aleatorias},
especialmente cuando las diferencias entre opciones no son tan claras.
Aquí entran:

\begin{center}\rule{0.5\linewidth}{0.5pt}\end{center}

\hypertarget{funciones-psicomuxe9tricas}{%
\subsubsection{\texorpdfstring{📊 \textbf{Funciones
psicométricas}}{📊 Funciones psicométricas}}\label{funciones-psicomuxe9tricas}}

Muestran cómo, cuando las diferencias entre estímulos (o Qs) son
pequeñas, las elecciones se vuelven menos consistentes. Se derivan dos
modelos:

\begin{center}\rule{0.5\linewidth}{0.5pt}\end{center}

\hypertarget{modelos-de-utilidad-aleatoria}{%
\subsubsection{\texorpdfstring{🧩 \textbf{Modelos de Utilidad
Aleatoria}}{🧩 Modelos de Utilidad Aleatoria}}\label{modelos-de-utilidad-aleatoria}}

\hypertarget{funciuxf3n-probit}{%
\paragraph{\texorpdfstring{✅ \textbf{Función
Probit}}{✅ Función Probit}}\label{funciuxf3n-probit}}

\begin{itemize}
\item
\item
  Supone que Q es \textbf{una variable aleatoria} (como una sensación).
\item
\item
  Usa la \textbf{distribución normal acumulada} para calcular la
  probabilidad de elegir una opción.
\item
\end{itemize}

\hypertarget{funciuxf3n-loguxedstica-logit-o-softmax}{%
\paragraph{\texorpdfstring{✅ \textbf{Función Logística (Logit o
Softmax)}}{✅ Función Logística (Logit o Softmax)}}\label{funciuxf3n-loguxedstica-logit-o-softmax}}

\begin{itemize}
\item
\item
  Más usada en aprendizaje automático.
\item
\item
  Transforma la diferencia entre Qs usando la función logística:

  P(a1)=11+e−λ(Qa1−Qa2)P(a\_1) =
  \frac{1}{1 + e^{-\lambda(Q_{a1} - Q_{a2})}}P(a1\hspace{0pt})=1+e−λ(Qa1\hspace{0pt}−Qa2\hspace{0pt})1\hspace{0pt}
\item
\item
  El parámetro \textbf{λ} controla la sensibilidad a las diferencias en
  Q.

  \begin{itemize}
  \item
  \item
    λ → 0 = elecciones casi aleatorias.
  \item
  \item
    λ → ∞ = elecciones deterministas (como la greedy).
  \item
  \end{itemize}
\item
\end{itemize}

\begin{center}\rule{0.5\linewidth}{0.5pt}\end{center}

\hypertarget{modelo-de-luce-elecciuxf3n-proporcional}{%
\subsubsection{\texorpdfstring{💖 \textbf{Modelo de Luce (Elección
Proporcional)}}{💖 Modelo de Luce (Elección Proporcional)}}\label{modelo-de-luce-elecciuxf3n-proporcional}}

\begin{itemize}
\item
\item
  No considera que Q sea aleatorio.
\item
\item
  Elige acciones \textbf{proporcionalmente a su valor Q}:

  P(a1)=Q(a1)Q(a1)+Q(a2)P(a\_1) =
  \frac{Q(a_1)}{Q(a_1) + Q(a_2)}P(a1\hspace{0pt})=Q(a1\hspace{0pt})+Q(a2\hspace{0pt})Q(a1\hspace{0pt})\hspace{0pt}

  Tiene una versión con λ que modula la sensibilidad, igual que el
  softmax.
\item
\end{itemize}

\begin{center}\rule{0.5\linewidth}{0.5pt}\end{center}

\hypertarget{en-resumen}{%
\subsubsection{✨ En resumen:}\label{en-resumen}}

\begin{longtable}[]{@{}
  >{\raggedright\arraybackslash}p{(\columnwidth - 2\tabcolsep) * \real{0.4324}}
  >{\raggedright\arraybackslash}p{(\columnwidth - 2\tabcolsep) * \real{0.5676}}@{}}
\toprule()
\begin{minipage}[b]{\linewidth}\raggedright
Concepto
\end{minipage} & \begin{minipage}[b]{\linewidth}\raggedright
¿Qué hace?
\end{minipage} \\
\midrule()
\endhead
\textbf{Política} & Decide qué acción tomar. \\
\textbf{Exploración vs Explotación} & Dilema sobre probar o repetir
acciones. \\
\textbf{Greedy (Maximización)} & Siempre elige la mejor acción (según
Q). \\
\textbf{ε-codicioso} & Mayoría greedy, pero a veces explora. \\
\textbf{Probit y Logit} & Modelos probabilísticos de elección. \\
\textbf{Softmax} & Regla logística para múltiples opciones. \\
\textbf{Luce (Acción proporcional)} & Elige basado en proporción del
valor Q. \\
\bottomrule()
\end{longtable}

\begin{center}\rule{0.5\linewidth}{0.5pt}\end{center}



\end{document}
